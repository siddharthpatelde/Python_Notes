\documentclass[a4paper,11pt]{article}

% ───────────────────────────
%         PACKAGES
% ───────────────────────────
\usepackage[utf8]{inputenc}
\usepackage[T1]{fontenc}
\usepackage{geometry}
\usepackage{graphicx}
\usepackage{amsmath, amssymb}
\usepackage{booktabs}
\usepackage{enumitem}
\usepackage{hyperref}
\usepackage{caption}
\usepackage{subcaption}

\usepackage{fancyhdr}
\usepackage{titlesec}
\usepackage{parskip}

\usepackage{minted}
\usepackage{inconsolata}  % clean monospaced font
\usepackage{mathpazo}     % professional text font

\usepackage{array}
\usepackage{booktabs}
\usepackage[table]{xcolor}


% ───────────────────────────
%         GEOMETRY
% ───────────────────────────
\geometry{
	a4paper,
	left=2.5cm,
	right=2.5cm,
	top=2.5cm,
	bottom=2.5cm
}

% ───────────────────────────
%         HEADER/FOOTER
% ───────────────────────────
\pagestyle{fancy}
\fancyhf{}
\renewcommand{\headrulewidth}{0.4pt}
\renewcommand{\footrulewidth}{0.4pt}
\lhead{\textsf{\textcolor{gray}{Python Notes}}}
\rhead{\textsf{\textcolor{gray}{Siddharth Patel}}}
\rfoot{\textsf{\textcolor{gray}{\thepage}}}

% ───────────────────────────
%         TITLE FORMAT
% ───────────────────────────
\titleformat{\section}
{\Large\bfseries\sffamily\color{black}}
{\thesection}{1em}{}

\titleformat{\subsection}
{\large\bfseries\sffamily\color{black}}
{\thesubsection}{1em}{}

% ───────────────────────────
%         MINTED SETTINGS
% ───────────────────────────
\definecolor{mintbg}{rgb}{0.97,0.97,0.97}
\setminted{
	bgcolor=mintbg,
	fontsize=\footnotesize,
	linenos=true,
	numbersep=5pt,
	frame=lines,
	framesep=2mm,
	breaklines=true,
	tabsize=4,
	encoding=utf8
}



% ───────────────────────────
%         TITLE PAGE
% ───────────────────────────
\title{\Huge \textbf{My Python Course Notes}\\[1ex]
	\Large Structured Revision for Every Lesson}
\author{\textbf{Siddharth Patel} \\ HTW Berlin}
\date{\today}

% ───────────────────────────
%         BEGIN DOCUMENT
% ───────────────────────────
\begin{document}
	
	\maketitle
	\thispagestyle{empty}
	\newpage
	\tableofcontents
	\newpage
	
	\section{If Else – Decision Making in Python}
	
	\subsection*{Q1: Greeting by First Name}
	Write a program that asks the user for their first name. If the user's name is \texttt{Marie} or \texttt{Otto}, the program should print \texttt{Hello Marie!} or \texttt{Hello Otto!} respectively. Otherwise, nothing should be printed. Leading and trailing spaces should be ignored. The input and output must match the following examples exactly.
	
	\textbf{Example 1:}\\
	\texttt{First name: \textcolor{blue}{\textit{Marie}}}\\
	\texttt{Hello Marie!}
	
	\textbf{Example 2:}\\
	\texttt{First name: \textcolor{blue}{\textit{Karl}}}\\
	(No output expected)
	
		\subsection*{Q1 – Solution}
		\inputminted{python}{Files/1/1.py}
	
	\subsection*{Q2: Compare Two Floating-Point Numbers}
	Write a program that asks the user to input two floating-point numbers, \texttt{a} and \texttt{b}. Use an \texttt{if-else} statement to determine and print the larger number.
	
	\textbf{Example:}\\
	\texttt{a = \textcolor{blue}{\textit{10}}}\\
	\texttt{b = \textcolor{blue}{\textit{11.2}}}\\
	\texttt{The larger number is: 11.2}
	
		
	\subsection*{Q2 – Solution}
		\inputminted{python}{Files/1/2.py}
	
	\subsection*{Q3: Compare with a Single if Statement}
	Solve the previous task again, but this time using only a single \texttt{if} statement without \texttt{else}. The input and output must match the example exactly.
	
	\textbf{Example:}\\
	\texttt{a = \textcolor{blue}{\textit{10}}}\\
	\texttt{b = \textcolor{blue}{\textit{11.2}}}\\
	\texttt{The larger number is: 11.2}
	
		\subsection*{Q3 – Solution}
	\inputminted{python}{Files/1/3.py}
	
	
	\subsection*{Q4: Largest of Three Numbers}
	Write a program that asks the user to input three floating-point numbers: \texttt{a}, \texttt{b}, and \texttt{c}. Then determine and print the largest number.
	
	\textbf{Example:}\\
	\texttt{a = \textcolor{blue}{\textit{10}}}\\
	\texttt{b = \textcolor{blue}{\textit{11.2}}}\\
	\texttt{c = \textcolor{blue}{\textit{-12.31}}}\\
	\texttt{The largest number is: 11.2}
	
		\subsection*{Q4 – Solution}
		\inputminted{python}{Files/1/4.py}
	
	\subsection*{Q5: Age-Based Classification (if-elif-else)}
	Write a program that asks the user for their age and prints a message based on the following rules:
	
	\begin{itemize}
		\item Under 18: \texttt{You are a minor.}
		\item Between 18 and 65: \texttt{You are an adult.}
		\item 66 and older: \texttt{You are of retirement age.}
	\end{itemize}
	
	\textbf{Example 1:}\\
	\texttt{Please enter your age: \textcolor{blue}{\textit{23}}}\\
	\texttt{You are an adult.}
	
	\textbf{Example 2:}\\
	\texttt{Please enter your age: \textcolor{blue}{\textit{13}}}\\
	\texttt{You are a minor.}
	
		
	\subsection*{Q5 – Solution}
		\inputminted{python}{Files/1/5.py}
	
	\subsection*{Q6: Age-Based Classification with Extended Range}
	Write a program similar to Q5 but with the following additional condition:
	
	\begin{itemize}
		\item Under 18: \texttt{You are a minor.}
		\item Between 18 and 65: \texttt{You are an adult.}
		\item Between 66 and 99: \texttt{You are of retirement age.}
		\item 100 or older: \texttt{You are in the Methuselah age group.}
	\end{itemize}
	
	\textbf{Example:}\\
	\texttt{Please enter your age: \textcolor{blue}{\textit{101}}}\\
	\texttt{You are in the Methuselah age group.}
	
		
	\subsection*{Q6 – Solution}
		\inputminted{python}{Files/1/6.py}
	
	\subsection*{Q7: Min and Max from Input Without Using min/max}
	Write a program that asks the user to enter a list of integers separated by commas. Then determine and print the smallest and largest number without using the built-in \texttt{min()} or \texttt{max()} functions.
	
	\textbf{Example:}\\
	\texttt{Enter numbers: \textcolor{blue}{\textit{1, 3, 7, -4, 9, 0, 2}}}\\
	\texttt{Smallest number: -4}\\
	\texttt{Largest number: 9}
	
		
	\subsection*{Q7 – Solution}
		\inputminted{python}{Files/1/7.py}
		
	\section{Match Statement – Pattern Matching in Python}
	
	\subsection*{Q1: Weekday Abbreviation to Full Name}
	Write a program that asks the user to input a weekday abbreviation (2 letters), such as \texttt{Mo}, \texttt{Di}, \texttt{Mi}, etc. The program should return the full weekday name. If the input is invalid, print \texttt{Error}. Use a \texttt{match} statement to solve this.
	
	\textbf{Example 1:}\\
	\texttt{Input: \textcolor{blue}{\textit{Mo}}}\\
	\texttt{Monday}
	
	\textbf{Example 2:}\\
	\texttt{Input: \textcolor{blue}{\textit{Mon}}}\\
	\texttt{Error}
	
		\subsection*{Q1 – Solution}
	\inputminted{python}{Files/2/1.py}
	
	\subsection*{Q2: Month Abbreviation to Season}
	Write a program that asks the user to enter a three-letter month abbreviation (e.g., \texttt{Jan}, \texttt{Feb}). The input should be case-insensitive. The program should return the corresponding season:
	
	\begin{itemize}
		\item Spring: Mar–May
		\item Summer: Jun–Aug
		\item Autumn: Sep–Nov
		\item Winter: Dec–Feb
	\end{itemize}
	
	Invalid input should return \texttt{Error}. Use a \texttt{match} statement with as few \texttt{case} blocks as possible (5 total).
	
	\textbf{Example:}\\
	\texttt{Input: \textcolor{blue}{\textit{Feb}}}\\
	\texttt{Winter}
	
		\subsection*{Q2 – Solution}
	\inputminted{python}{Files/2/2.py}
	
	
	\subsection*{Q3: Rewrite if-elif-else Using match}
	Given the following code:
	
	\begin{minted}{python}
		val = int(input('Value: '))
		if val >= 0 and val < 4:
		print('Small number')
		elif val < 6:
		print('Medium number')
		elif val == 7 or val == 9:
		print('Special number')
		else:
		print('Some other number')
	\end{minted}
	
	Rewrite the program using a \texttt{match} statement instead of \texttt{if-elif-else} to achieve the same logic and output.
	

	\subsection*{Q3 – Solution}
	\inputminted{python}{Files/2/3.py}
	
	
	\section{The \texttt{for} Loop – Counting, Iteration, and Containers}
	
	\subsection*{Q1: Count from 1 to a User-Defined Value}
	Write a program that asks the user to input a number. Then use a \texttt{for} loop to print all numbers from 1 to the entered value.
	
	\textbf{Example:}\\
	\texttt{Last count value: \textcolor{blue}{\textit{4}}}\\
	\texttt{1\\2\\3\\4}
	
		\subsection*{Q1 – Solution}
	\inputminted{python}{Files/3/1.py}
	
	
	\subsection*{Q2: Count Down to 1}
	Repeat the previous task, but count down from the input value to 1, using a step of -1.
	
	\textbf{Example:}\\
	\texttt{Last count value: \textcolor{blue}{\textit{4}}}\\
	\texttt{4\\3\\2\\1}
	
	
	\subsection*{Q2 – Solution}
	\inputminted{python}{Files/3/2.py}
	
	\subsection*{Q3: Factorial with a \texttt{for} Loop}
	Write a program that calculates the factorial of a user-given number \texttt{n}. Use a \texttt{for} loop starting at 1.
	
	\textbf{Example:}\\
	\texttt{Please enter n: \textcolor{blue}{\textit{4}}}\\
	\texttt{n! = 24}
	
		
	\subsection*{Q3 – Solution}
	\inputminted{python}{Files/3/3.py}
	
	\subsection*{Q4: Count Occurrences of a Character in a String}
	Write a program that asks the user for a string and counts how many times the letter \texttt{e} appears using a \texttt{for} loop.
	
	\textbf{Example:}\\
	\texttt{Input: \textcolor{blue}{\textit{Das Leben ist schoen!}}}\\
	\texttt{Number of e: 3}
	
		
	\subsection*{Q4 – Solution}
	\inputminted{python}{Files/3/4.py}
	
	\subsection*{Q5: Sum All Elements in a List}
	Given the list:
	
	\texttt{intlist = [1, 4, 3, 7, -5, 3, 5]}
	
	Write a program that uses a \texttt{for} loop to sum all elements and print the result.
	
	\textbf{Output:}\\
	\texttt{Sum: 18}
	

	
	\subsection*{Q5 – Solution}
	\inputminted{python}{Files/3/5.py}
	
		
	


	





	
	
\end{document}

