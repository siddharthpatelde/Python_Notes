\documentclass[a4paper,11pt]{article}

% ───────────────────────────
%         PACKAGES
% ───────────────────────────
\usepackage[utf8]{inputenc}
\usepackage[T1]{fontenc}
\usepackage{geometry}
\usepackage{graphicx}
\usepackage{amsmath, amssymb}
\usepackage{booktabs}
\usepackage{enumitem}
\usepackage{hyperref}
\usepackage{caption}
\usepackage{subcaption}

\usepackage{fancyhdr}
\usepackage{titlesec}
\usepackage{parskip}

\usepackage{minted}
\usepackage{inconsolata}  % clean monospaced font
\usepackage{mathpazo}     % professional text font

\usepackage{array}
\usepackage{booktabs}
\usepackage[table]{xcolor}


% ───────────────────────────
%         GEOMETRY
% ───────────────────────────
\geometry{
	a4paper,
	left=2.5cm,
	right=2.5cm,
	top=2.5cm,
	bottom=2.5cm
}

% ───────────────────────────
%         HEADER/FOOTER
% ───────────────────────────
\pagestyle{fancy}
\fancyhf{}
\renewcommand{\headrulewidth}{0.4pt}
\renewcommand{\footrulewidth}{0.4pt}
\lhead{\textsf{\textcolor{gray}{Python Notes}}}
\rhead{\textsf{\textcolor{gray}{Siddharth Patel}}}
\rfoot{\textsf{\textcolor{gray}{\thepage}}}

% ───────────────────────────
%         TITLE FORMAT
% ───────────────────────────
\titleformat{\section}
{\Large\bfseries\sffamily\color{black}}
{\thesection}{1em}{}

\titleformat{\subsection}
{\large\bfseries\sffamily\color{black}}
{\thesubsection}{1em}{}

% ───────────────────────────
%         MINTED SETTINGS
% ───────────────────────────
\definecolor{mintbg}{rgb}{0.97,0.97,0.97}
\setminted{
	bgcolor=mintbg,
	fontsize=\footnotesize,
	linenos=true,
	numbersep=5pt,
	frame=lines,
	framesep=2mm,
	breaklines=true,
	tabsize=4,
	encoding=utf8
}



% ───────────────────────────
%         TITLE PAGE
% ───────────────────────────
\title{\Huge \textbf{My Python Course Notes}\\[1ex]
	\Large Python Exercises and Examples}
\author{\textbf{Siddharth Patel} \\ HTW Berlin}
\date{\today}

% ───────────────────────────
%         BEGIN DOCUMENT
% ───────────────────────────
\begin{document}
	
	\maketitle
	\thispagestyle{empty}
	\newpage
	\tableofcontents
	\newpage

	
	\section{If Else – Decision Making in Python}
	
	\subsection*{Q1: Greeting by First Name}
	Write a program that asks the user for their first name. If the user's name is \texttt{Marie} or \texttt{Otto}, the program should print \texttt{Hello Marie!} or \texttt{Hello Otto!} respectively. Otherwise, nothing should be printed. Leading and trailing spaces should be ignored. The input and output must match the following examples exactly.
	
	\textbf{Example 1:}\\
	\texttt{First name: \textcolor{blue}{\textit{Marie}}}\\
	\texttt{Hello Marie!}
	
	\textbf{Example 2:}\\
	\texttt{First name: \textcolor{blue}{\textit{Karl}}}\\
	(No output expected)
	
		\subsection*{Q1 – Solution}
		\inputminted{python}{Files/1/1.py}
	
	\subsection*{Q2: Compare Two Floating-Point Numbers}
	Write a program that asks the user to input two floating-point numbers, \texttt{a} and \texttt{b}. Use an \texttt{if-else} statement to determine and print the larger number.
	
	\textbf{Example:}\\
	\texttt{a = \textcolor{blue}{\textit{10}}}\\
	\texttt{b = \textcolor{blue}{\textit{11.2}}}\\
	\texttt{The larger number is: 11.2}
	
		
	\subsection*{Q2 – Solution}
		\inputminted{python}{Files/1/2.py}
	
	\subsection*{Q3: Compare with a Single if Statement}
	Solve the previous task again, but this time using only a single \texttt{if} statement without \texttt{else}. The input and output must match the example exactly.
	
	\textbf{Example:}\\
	\texttt{a = \textcolor{blue}{\textit{10}}}\\
	\texttt{b = \textcolor{blue}{\textit{11.2}}}\\
	\texttt{The larger number is: 11.2}
	
		\subsection*{Q3 – Solution}
	\inputminted{python}{Files/1/3.py}
	
	
	\subsection*{Q4: Largest of Three Numbers}
	Write a program that asks the user to input three floating-point numbers: \texttt{a}, \texttt{b}, and \texttt{c}. Then determine and print the largest number.
	
	\textbf{Example:}\\
	\texttt{a = \textcolor{blue}{\textit{10}}}\\
	\texttt{b = \textcolor{blue}{\textit{11.2}}}\\
	\texttt{c = \textcolor{blue}{\textit{-12.31}}}\\
	\texttt{The largest number is: 11.2}
	
		\subsection*{Q4 – Solution}
		\inputminted{python}{Files/1/4.py}
	
	\subsection*{Q5: Age-Based Classification (if-elif-else)}
	Write a program that asks the user for their age and prints a message based on the following rules:
	
	\begin{itemize}
		\item Under 18: \texttt{You are a minor.}
		\item Between 18 and 65: \texttt{You are an adult.}
		\item 66 and older: \texttt{You are of retirement age.}
	\end{itemize}
	
	\textbf{Example 1:}\\
	\texttt{Please enter your age: \textcolor{blue}{\textit{23}}}\\
	\texttt{You are an adult.}
	
	\textbf{Example 2:}\\
	\texttt{Please enter your age: \textcolor{blue}{\textit{13}}}\\
	\texttt{You are a minor.}
	
		
	\subsection*{Q5 – Solution}
		\inputminted{python}{Files/1/5.py}
	
	\subsection*{Q6: Age-Based Classification with Extended Range}
	Write a program similar to Q5 but with the following additional condition:
	
	\begin{itemize}
		\item Under 18: \texttt{You are a minor.}
		\item Between 18 and 65: \texttt{You are an adult.}
		\item Between 66 and 99: \texttt{You are of retirement age.}
		\item 100 or older: \texttt{You are in the Methuselah age group.}
	\end{itemize}
	
	\textbf{Example:}\\
	\texttt{Please enter your age: \textcolor{blue}{\textit{101}}}\\
	\texttt{You are in the Methuselah age group.}
	
		
	\subsection*{Q6 – Solution}
		\inputminted{python}{Files/1/6.py}
	
	\subsection*{Q7: Min and Max from Input Without Using min/max}
	Write a program that asks the user to enter a list of integers separated by commas. Then determine and print the smallest and largest number without using the built-in \texttt{min()} or \texttt{max()} functions.
	
	\textbf{Example:}\\
	\texttt{Enter numbers: \textcolor{blue}{\textit{1, 3, 7, -4, 9, 0, 2}}}\\
	\texttt{Smallest number: -4}\\
	\texttt{Largest number: 9}
	
		
	\subsection*{Q7 – Solution}
		\inputminted{python}{Files/1/7.py}
		
		\newpage
		
	\section{Match Statement – Pattern Matching in Python}
	
	\subsection*{Q1: Weekday Abbreviation to Full Name}
	Write a program that asks the user to input a weekday abbreviation (2 letters), such as \texttt{Mo}, \texttt{Di}, \texttt{Mi}, etc. The program should return the full weekday name. If the input is invalid, print \texttt{Error}. Use a \texttt{match} statement to solve this.
	
	\textbf{Example 1:}\\
	\texttt{Input: \textcolor{blue}{\textit{Mo}}}\\
	\texttt{Monday}
	
	\textbf{Example 2:}\\
	\texttt{Input: \textcolor{blue}{\textit{Mon}}}\\
	\texttt{Error}
	
		\subsection*{Q1 – Solution}
	\inputminted{python}{Files/2/1.py}
	
	\subsection*{Q2: Month Abbreviation to Season}
	Write a program that asks the user to enter a three-letter month abbreviation (e.g., \texttt{Jan}, \texttt{Feb}). The input should be case-insensitive. The program should return the corresponding season:
	
	\begin{itemize}
		\item Spring: Mar–May
		\item Summer: Jun–Aug
		\item Autumn: Sep–Nov
		\item Winter: Dec–Feb
	\end{itemize}
	
	Invalid input should return \texttt{Error}. Use a \texttt{match} statement with as few \texttt{case} blocks as possible (5 total).
	
	\textbf{Example:}\\
	\texttt{Input: \textcolor{blue}{\textit{Feb}}}\\
	\texttt{Winter}
	
		\subsection*{Q2 – Solution}
	\inputminted{python}{Files/2/2.py}
	
	
	\subsection*{Q3: Rewrite if-elif-else Using match}
	Given the following code:
	
	\begin{minted}{python}
		val = int(input('Value: '))
		if val >= 0 and val < 4:
		print('Small number')
		elif val < 6:
		print('Medium number')
		elif val == 7 or val == 9:
		print('Special number')
		else:
		print('Some other number')
	\end{minted}
	
	Rewrite the program using a \texttt{match} statement instead of \texttt{if-elif-else} to achieve the same logic and output.
	

	\subsection*{Q3 – Solution}
	\inputminted{python}{Files/2/3.py}
	
	\newpage
	
	\section{The \texttt{for} Loop – Counting, Iteration, and Containers}
	
	\subsection*{Q1: Count from 1 to a User-Defined Value}
	Write a program that asks the user to input a number. Then use a \texttt{for} loop to print all numbers from 1 to the entered value.
	
	\textbf{Example:}\\
	\texttt{Last count value: \textcolor{blue}{\textit{4}}}\\
	\texttt{1\\2\\3\\4}
	
		\subsection*{Q1 – Solution}
	\inputminted{python}{Files/3/1.py}
	
	
	\subsection*{Q2: Count Down to 1}
	Repeat the previous task, but count down from the input value to 1, using a step of -1.
	
	\textbf{Example:}\\
	\texttt{Last count value: \textcolor{blue}{\textit{4}}}\\
	\texttt{4\\3\\2\\1}
	
	
	\subsection*{Q2 – Solution}
	\inputminted{python}{Files/3/2.py}
	
	\subsection*{Q3: Factorial with a \texttt{for} Loop}
	Write a program that calculates the factorial of a user-given number \texttt{n}. Use a \texttt{for} loop starting at 1.
	
	\textbf{Example:}\\
	\texttt{Please enter n: \textcolor{blue}{\textit{4}}}\\
	\texttt{n! = 24}
	
		
	\subsection*{Q3 – Solution}
	\inputminted{python}{Files/3/3.py}
	
	\subsection*{Q4: Count Occurrences of a Character in a String}
	Write a program that asks the user for a string and counts how many times the letter \texttt{e} appears using a \texttt{for} loop.
	
	\textbf{Example:}\\
	\texttt{Input: \textcolor{blue}{\textit{Das Leben ist schoen!}}}\\
	\texttt{Number of e: 3}
	
		
	\subsection*{Q4 – Solution}
	\inputminted{python}{Files/3/4.py}
	
	\subsection*{Q5: Sum All Elements in a List}
	Given the list:
	
	\texttt{intlist = [1, 4, 3, 7, -5, 3, 5]}
	
	Write a program that uses a \texttt{for} loop to sum all elements and print the result.
	
	\textbf{Output:}\\
	\texttt{Sum: 18}
	

	
	\subsection*{Q5 – Solution}
	\inputminted{python}{Files/3/5.py}
	
		
	
\newpage

\section{Functions – Parameters, Return Values, and Reusability}

\subsection*{Q1: Average of Three Numbers}
Write a function that accepts three floating-point numbers as parameters and returns their average. Add a main program that asks the user to enter three values and then calls the function and displays the result.

\textbf{Example:}\\
\texttt{a = \textcolor{blue}{\textit{12}}}\\
\texttt{b = \textcolor{blue}{\textit{18}}}\\
\texttt{c = \textcolor{blue}{\textit{6}}}\\
\texttt{Average = 12}

\subsection*{Q1 – Solution}
\inputminted{python}{Files/4/1.py}

\subsection*{Q2: Factorial Function}
Write a function that receives an integer \texttt{n} as a parameter and returns \texttt{n!}. Add a main program that gets input from the user and prints the result.

\textbf{Example:}\\
\texttt{n = \textcolor{blue}{\textit{4}}}\\
\texttt{n! = 24}

\subsection*{Q2 – Solution}
\inputminted{python}{Files/4/2.py}

\subsection*{Q3: Area and Perimeter of Rectangle}
Write a function \texttt{rect(a, b)} that receives two sides of a rectangle and returns the area and perimeter. Add a main program that takes inputs and prints both results.

\textbf{Example:}\\
\texttt{a = \textcolor{blue}{\textit{4}}}\\
\texttt{b = \textcolor{blue}{\textit{3}}}\\
\texttt{Area = 12}\\
\texttt{Perimeter = 14}

\subsection*{Q3 – Solution}
\inputminted{python}{Files/4/3.py}

\subsection*{Q4: Count Vowels Using Function}
Write a function that accepts a string \texttt{s} and a letter \texttt{letter}, and returns how often that letter appears (case-insensitive). In the main program, prompt for a string and print the number of occurrences of \texttt{a, e, i, o, u}.

\subsection*{Q4 – Solution}
\inputminted{python}{Files/4/4.py}

\textbf{Example:}\\
\texttt{Input: \textcolor{blue}{\textit{Die Summe der Kathetenquadrate ist Gleich dem Hypotenusenquadrat}}}\\
\texttt{a: 5\\e: 10\\i: 3\\o: 1\\u: 4}

\subsection*{Q5: Custom \texttt{join()} Implementation}
Write a function \texttt{myJoin(lst, sep)} that replicates the behavior of \texttt{join()} without actually using it. It should combine list elements into a single string, separated by \texttt{sep}.

\textbf{Test Code:}
\begin{minted}{python}
	mylist = ['Wir', 'müssen', 'uns', 'Sisyphos', 'als', 'einen', 'glücklichen', 'Menschen', 'vorstellen']
	mystring = myJoin(mylist, '-')
	print(mystring)
\end{minted}

\textbf{Expected Output:}\\
\texttt{Wir-müssen-uns-Sisyphos-als-einen-glücklichen-Menschen-vorstellen}

\subsection*{Q5 – Solution}
\inputminted{python}{Files/4/5.py}


\subsection*{Q6: Recursive Sum Function}
Write a recursive function \texttt{total(n)} that returns the sum from 1 to \texttt{n} without using loops. Include a main program to prompt the user and show the result.

\textbf{Example:}\\
\texttt{n = \textcolor{blue}{\textit{4}}}\\
\texttt{Sum = 10}




\subsection*{Q6 – Solution}
\inputminted{python}{Files/4/6.py}

\newpage
\section{The \texttt{while} Loop – Manual Iteration and Termination Conditions}

\subsection*{Q1: Rewrite a \texttt{for}-loop as a \texttt{while}-loop (0 to 4)}
Given the following \texttt{for}-loop:

\begin{minted}{python}
	for i in range(5):
	print(i)
\end{minted}

Rewrite it using a \texttt{while}-loop to produce the same output:

\textbf{Expected Output:}\\
\texttt{0\\1\\2\\3\\4}

\subsection*{Q1 – Solution}
\inputminted{python}{Files/5/1.py}


\subsection*{Q2: Rewrite a \texttt{for}-loop with start, stop, step}
Given the following \texttt{for}-loop:

\begin{minted}{python}
	for i in range(5, 13, 2):
	print(i)
\end{minted}

Rewrite it using a \texttt{while}-loop to produce the same output:

\textbf{Expected Output:}\\
\texttt{5\\7\\9\\11}


\subsection*{Q2 – Solution}
\inputminted{python}{Files/5/2.py}

\subsection*{Q3: Capitalize User Input Until “Ende”}
Write a program that reads user input, converts it to uppercase, and prints it. The loop should end when the user enters \texttt{ende} (case-insensitive). Use a \texttt{while}-loop.

\textbf{Example:}
\begin{flushleft}
	\texttt{Input: \textcolor{blue}{\textit{hallo welt}}}\\
	\texttt{Output: HALLO WELT}\\
	\texttt{Input: \textcolor{blue}{\textit{Noch ein Test}}}\\
	\texttt{Output: NOCH EIN TEST}\\
	\texttt{Input: \textcolor{blue}{\textit{Ende}}}
\end{flushleft}


\subsection*{Q3 – Solution}
\inputminted{python}{Files/5/3.py}

\subsection*{Q4: Sum Until Empty Input}
Write a program that asks the user to input integers, one at a time, inside a \texttt{while}-loop. When the user enters an empty line, stop and display the sum. Ignore extra spaces.

\textbf{Example:}
\begin{flushleft}
	\texttt{Input: \textcolor{blue}{\textit{1}}}\\
	\texttt{Input: \textcolor{blue}{\textit{   55}}}\\
	\texttt{Input: \textcolor{blue}{\textit{7}}}\\
	\texttt{Input: \textcolor{blue}{\textit{}}}\\
	\texttt{Sum: 63}
\end{flushleft}

\subsection*{Q4 – Solution}
\inputminted{python}{Files/5/4.py}

\subsection*{Q5: Sum from Multi-Value Inputs with Nested Loops}
Write a program that allows the user to input multiple integers in a single line (space-separated). Continue until an empty line is entered, then print the total sum. Use only \texttt{while}-loops.

\textbf{Example:}
\begin{flushleft}
	\texttt{Input: \textcolor{blue}{\textit{1 3 5}}}\\
	\texttt{Input: \textcolor{blue}{\textit{   7   8}}}\\
	\texttt{Input: \textcolor{blue}{\textit{4}}}\\
	\texttt{Input: \textcolor{blue}{\textit{}}}\\
	\texttt{Sum: 28}
\end{flushleft}


\subsection*{Q5 – Solution}
\inputminted{python}{Files/5/5.py}

\newpage
\section{Importing Modules and Libraries – Using Built-in Functionality}

\subsection*{Q1: Simulating Dice Rolls with \texttt{random.randint()}}
Write a program that asks the user how many times a die should be rolled. Then generate that many random integers between 1 and 6 using \texttt{random.randint()}. Make sure to call \texttt{random.seed(0)} at the beginning so that the sequence is repeatable.

\textbf{Example:}
\begin{flushleft}
	\texttt{Rolls: \textcolor{blue}{\textit{3}}}\\
	\texttt{4\\4\\1}
\end{flushleft}

\subsection*{Q1 – Solution}
\inputminted{python}{Files/6/1.py}

\subsection*{Q2: Random Word Selection Using \texttt{from ... import}}
Write a program that imports only \texttt{seed()} and \texttt{choice()} from the \texttt{random} module. Use \texttt{seed(0)}, then ask the user to enter a sentence. Split the sentence into words and use \texttt{choice()} to randomly pick one word to display.

\textbf{Example:}
\begin{flushleft}
	\texttt{Input: \textcolor{blue}{\textit{red yellow blue}}}\\
	\texttt{blue}
\end{flushleft}

\subsection*{Q2 – Solution}
\inputminted{python}{Files/6/2.py}

\subsection*{Q3: Full Access to All \texttt{random} Functions}
Given the following main program:

\begin{minted}{python}
	seed(0)
	print(randint(1, 10))
	print(randrange(1, 10))
	print(random())
	print(choice(['red', 'yellow', 'green']))
\end{minted}

Complete the code so that all \texttt{random} functions are accessible directly, without using the \texttt{random.} prefix.


\subsection*{Q3 – Solution}
\inputminted{python}{Files/6/3.py}
\newpage

\section{Lists and Tuples – Access, Operations, and Manual Algorithms}

\subsection*{Q1: Accessing List Elements}
Given the list:

\begin{minted}{python}
	lst = [1, 3, 9, -11, 2, 0, 21]
\end{minted}

Access and print the second and last element of the list. Then create a new list containing the 3rd, 4th, and 5th elements and print it.

\textbf{Expected Output:}
\begin{flushleft}
	\texttt{3}\\
	\texttt{21}\\
	\texttt{[9, -11, 2]}
\end{flushleft}

\subsection*{Q1 – Solution}
\inputminted{python}{Files/7/1.py}

\subsection*{Q2: Element-wise Addition of Equal-Length Lists}
Given:

\begin{minted}{python}
	lst1 = [1, 3, 9, -11, 2, 0, 21]
	lst2 = [2, 1, -2, 0, 3, -2, -11]
\end{minted}

Write a program that adds the lists element by element and stores the result in a new list.

\textbf{Expected Output:}\\
\texttt{Result: [3, 4, 7, -11, 5, -2, 10]}

\subsection*{Q2 – Solution}
\inputminted{python}{Files/7/2.py}

\subsection*{Q3: Element-wise Addition with Unequal Lengths}
Now allow \texttt{lst1} and \texttt{lst2} to have different lengths. When that’s the case, append the extra elements from the longer list to the result.

\begin{minted}{python}
	lst1 = [1, 3, 9, -11, 2, 0, 21]
	lst2 = [2, 1, -2, 0, 3, -2, -11, 4, 8]
\end{minted}

\textbf{Expected Output:}\\
\texttt{Result: [3, 4, 7, -11, 5, -2, 10, 4, 8]}

\subsection*{Q3 – Solution}
\inputminted{python}{Files/7/3.py}

\subsection*{Q4: Convert User Input to List of Integers}
Prompt the user to enter numbers in a single line. Convert them into integers and store them in a list.

\textbf{Example:}
\begin{flushleft}
	\texttt{Input: \textcolor{blue}{\textit{1 2 3 4 5}}}\\
	\texttt{Result: [1, 2, 3, 4, 5]}
\end{flushleft}



\subsection*{Q4 – Solution}
\inputminted{python}{Files/7/4.py}

\subsection*{Q5: Sort List Without Using \texttt{sort()} or \texttt{sorted()}}
Given the list:

\begin{minted}{python}
	lst = [1, 3, 9, -11, 2, 0, 21]
\end{minted}

Write a program that sorts the list in ascending order using your own sorting algorithm (e.g., selection or bubble sort).

\textbf{Expected Output:}\\
\texttt{Result: [-11, 0, 1, 2, 3, 9, 21]}


\subsection*{Q5 – Solution}
\inputminted{python}{Files/7/5.py}

\newpage
	\section{Dictionaries – Key-Value Data Storage and Access}
	
	\subsection*{Q1: Loop Through Dictionary Keys and Values}
	Given the dictionary:
	
	\begin{minted}{python}
		tab = {'Mo': 1, 'Di': 2, 'Mi': 3, 'Do': 4, 'Fr': 5, 'Sa': 6, 'So': 7}
	\end{minted}
	
	Write a program that prints each key and its corresponding value on a separate line.
	
	\textbf{Expected Output:}
	\begin{flushleft}
		\texttt{Mo: 1\\Di: 2\\Mi: 3\\Do: 4\\Fr: 5\\Sa: 6\\So: 7}
	\end{flushleft}
	
	\subsection*{Q1 – Solution}
	\inputminted{python}{Files/8/1.py}
	
	\subsection*{Q2: Check If a Key Exists in Dictionary}
	Using the same dictionary \texttt{tab}, prompt the user for input. If the input is a key in the dictionary, print its value. Otherwise, print \texttt{"Unknown word"}.
	
	\textbf{Example 1:}
	\begin{flushleft}
		\texttt{Input: \textcolor{blue}{\textit{Mi}}}\\
		\texttt{3}
	\end{flushleft}
	
	\textbf{Example 2:}
	\begin{flushleft}
		\texttt{Input: \textcolor{blue}{\textit{Mittwoch}}}\\
		\texttt{Unknown word}
	\end{flushleft}
	
	
	\subsection*{Q2 – Solution}
	\inputminted{python}{Files/8/2.py}
	
	\subsection*{Q3: Count Character Frequency Using Dictionary}
	Prompt the user to enter a sentence. Use a dictionary to count the number of times each character appears. Output the dictionary.
	
	\textbf{Example:}
	\begin{flushleft}
		\texttt{Input: \textcolor{blue}{\textit{Eins und Eins ist Zwei}}}\\
		\texttt{Result: \{'E': 2, 'i': 4, 'n': 3, 's': 3, ' ': 4, 'u': 1, 'd': 1, 't': 1, 'Z': 1, 'w': 1, 'e': 1\}}
	\end{flushleft}
	
	\subsection*{Q3 – Solution}
	\inputminted{python}{Files/8/3.py}
	
	\subsection*{Q4: Interactive Dictionary with Update and Lookup}
	Write a program that starts with an empty dictionary \texttt{tab}. In a loop, prompt the user for input:
	
	\begin{itemize}
		\item If the input contains a single word and the key exists in \texttt{tab}, print the value.
		\item If the key doesn’t exist, print \texttt{"Unknown word"}.
		\item If the input contains two words, insert the first as key and second as value. Print \texttt{"New entry"}.
		\item If the input is \texttt{Ende}, stop and print the dictionary.
	\end{itemize}
	
	\textbf{Example:}
	\begin{flushleft}
		\texttt{Input: \textcolor{blue}{\textit{Hallo}}}\\
		\texttt{Unknown word}\\
		\texttt{Input: \textcolor{blue}{\textit{Hallo Welt}}}\\
		\texttt{New entry}\\
		\texttt{Input: \textcolor{blue}{\textit{Hallo}}}\\
		\texttt{Welt}\\
		\texttt{Input: \textcolor{blue}{\textit{Ende}}}\\
		\texttt{\{'Hallo': 'Welt'\}}
	\end{flushleft}
	
\subsection*{Q4 – Solution}
\inputminted{python}{Files/8/4.py}
\newpage
\section{Sets – Unique Collections and Set Operations}

\subsection*{Q1: Collect Unique Words Until “Ende”}
Write a program that asks the user to enter words in a loop. Store each word in a set. Stop when the user enters \texttt{Ende}. The word \texttt{Ende} should not be included in the set. After termination, print the resulting set.

\textbf{Example:}
\begin{flushleft}
	\texttt{Input: \textcolor{blue}{\textit{Hallo}}}\\
	\texttt{Input: \textcolor{blue}{\textit{Berlin}}}\\
	\texttt{Input: \textcolor{blue}{\textit{Hallo}}}\\
	\texttt{Input: \textcolor{blue}{\textit{Welt}}}\\
	\texttt{Input: \textcolor{blue}{\textit{Ende}}}\\
	\texttt{\{'Hallo', 'Welt', 'Berlin'\}}
\end{flushleft}

\subsection*{Q1 – Solution}
\inputminted{python}{Files/9/1.py}

\subsection*{Q2: Remove Words from a Set}
Given the set:

\begin{minted}{python}
	s = {'der', 'die', 'das', 'und', 'er', 'sie', 'es', 'du'}
\end{minted}

Prompt the user for words. If the word is in the set, remove it and print \texttt{Removed}. If it’s not in the set, print \texttt{Not found}. Stop on \texttt{Ende} and print the set.

\textbf{Example:}
\begin{flushleft}
	\texttt{Input: \textcolor{blue}{\textit{Hallo}}}\\
	\texttt{Not found}\\
	\texttt{Input: \textcolor{blue}{\textit{und}}}\\
	\texttt{Removed}\\
	\texttt{Input: \textcolor{blue}{\textit{und}}}\\
	\texttt{Not found}\\
	\texttt{Input: \textcolor{blue}{\textit{Ende}}}\\
	\texttt{\{'der', 'die', 'das', 'er', 'sie', 'es', 'du'\}}
\end{flushleft}

\subsection*{Q2 – Solution}
\inputminted{python}{Files/9/2.py}

\subsection*{Q3: Intersection of Word Sets from Two Sentences}
Prompt the user to enter two sentences. Split the sentences into words and store them in sets. Then compute and print the intersection.

\textbf{Example:}
\begin{flushleft}
	\texttt{Sentence 1: \textcolor{blue}{\textit{Der Löwe ist der Adler unter den Katzen}}}\\
	\texttt{Sentence 2: \textcolor{blue}{\textit{Der Adler ist der Löwen unter den Piematzen}}}\\
	\texttt{\{'der', 'unter', 'ist', 'Der', 'den', 'Adler'\}}
\end{flushleft}

\subsection*{Q3 – Solution}
\inputminted{python}{Files/9/3.py}

\subsection*{Q4: Union of Word Sets from Two Sentences}
Same as Q3, but compute the union of the sets instead.

\textbf{Example:}
\begin{flushleft}
	\texttt{Sentence 1: \textcolor{blue}{\textit{Der Löwe ist der Adler unter den Katzen}}}\\
	\texttt{Sentence 2: \textcolor{blue}{\textit{Der Adler ist der Löwen unter den Piematzen}}}\\
	\texttt{\{'Der', 'Katzen', 'der', 'Piematzen', 'Adler', 'den', 'Löwe', 'unter', 'Löwen', 'ist'\}}
\end{flushleft}

\subsection*{Q4 – Solution}
\inputminted{python}{Files/9/4.py}

\subsection*{Q5: Difference – Words in First Sentence Only}
Now compute the difference of the two sets: words that are in the first sentence but not in the second.

\textbf{Example:}
\begin{flushleft}
	\texttt{Sentence 1: \textcolor{blue}{\textit{Der Löwe ist der Adler unter den Katzen}}}\\
	\texttt{Sentence 2: \textcolor{blue}{\textit{Der Adler ist der Löwen unter den Piematzen}}}\\
	\texttt{\{'Katzen', 'Löwe'\}}
\end{flushleft}

\subsection*{Q5 – Solution}
\inputminted{python}{Files/9/5.py}
\newpage
\section{Classes and Objects – OOP with Methods, Attributes, and Inheritance}

\subsection*{Q1: Rectangle Class – Area and Perimeter}
Write a class \texttt{Rectangle} with two attributes \texttt{a} and \texttt{b} (side lengths). The constructor should initialize both attributes. Add methods \texttt{getArea()} and \texttt{getPerimeter()} that return the area and perimeter.

Add a main program that creates two rectangles with dimensions 2×3 and 4×7. Output should be:

\textbf{Expected Output:}
\begin{flushleft}
	\texttt{Rectangle 1 Area: 6.0}\\
	\texttt{Rectangle 1 Perimeter: 10.0}\\
	\texttt{Rectangle 2 Area: 28.0}\\
	\texttt{Rectangle 2 Perimeter: 22.0}
\end{flushleft}

\subsection*{Q1 – Solution}
\inputminted{python}{Files/10/1.py}

\subsection*{Q2: Circle Class – Radius and Area (with private attribute)}
Write a class \texttt{Circle} with a private attribute \texttt{\_r}. Include:

\begin{itemize}
	\item \texttt{getR()} – returns radius
	\item \texttt{setR(r)} – sets a new radius
	\item \texttt{getArea()} – returns area using $\pi = 3.14$
\end{itemize}

In the main program, create two circles with radii 3 and 4. Print radius and area. Then update circle1’s radius to 10 and print again.

\textbf{Expected Output:}
\begin{flushleft}
	\texttt{Circle 1: r = 3, A = 28.26}\\
	\texttt{Circle 2: r = 4, A = 50.24}\\
	\texttt{Circle 1: r = 10, A = 314.0}
\end{flushleft}

\subsection*{Q2 – Solution}
\inputminted{python}{Files/10/2.py}

\subsection*{Q3: Triangle Class with Magic Method}
Create a class \texttt{Triangle} with private attributes \texttt{\_c} (base) and \texttt{\_h} (height). Include:

\begin{itemize}
	\item Constructor to initialize both
	\item \texttt{getArea()} method for triangle area
	\item \texttt{\_\_str\_\_()} magic method to support \texttt{print(triangle)} with formatted output
\end{itemize}

\textbf{Main program:}
\begin{minted}{python}
	tri = Triangle(4, 5)
	print(tri)
\end{minted}

\textbf{Expected Output:}\\
\texttt{Base: 4, Height: 5, Area: 10}


\subsection*{Q3 – Solution}
\inputminted{python}{Files/10/3.py}

\subsection*{Q4: Inheritance – Tank, Cylinder, and Cuboid}
Write a base class \texttt{Tank} with private attributes:

\begin{itemize}
	\item \texttt{\_A} – area of the base
	\item \texttt{\_h} – height
\end{itemize}

Include:
\begin{itemize}
	\item \texttt{getVolume()} – returns \texttt{A × h}
	\item \texttt{getArea()} – returns \texttt{A}
\end{itemize}

Now define two subclasses:
- \texttt{Cilinder(r, h)}: calculates and passes $\pi r^2$ as base area; implements \texttt{getR()} using $\sqrt{A/\pi}$
- \texttt{Quarder(a, h)}: calculates and passes $a^2$ as base area; implements \texttt{getA()} using $\sqrt{A}$

\textbf{Main program:}
Create a \texttt{Cilinder} with \texttt{r = 1, h = 2} and a \texttt{Quarder} with \texttt{a = 3, h = 4}. Output:

\textbf{Expected Output:}
\begin{flushleft}
	\texttt{Cilinder: r = 1.0, Volume = 6.28}\\
	\texttt{Quarder: a = 3.0, Volume = 36.0}
\end{flushleft}


\subsection*{Q4 – Solution}
\inputminted{python}{Files/10/4.py}

\newpage
\section{Comprehensions – Lists, Dictionaries, and Sets in One Line}

\subsection*{Q1: List of Squares up to N}
Prompt the user to enter a number \texttt{N}. Use a list comprehension to generate a list of square numbers from 0 to \texttt{N}.

\textbf{Example:}
\begin{flushleft}
	\texttt{N = \textcolor{blue}{\textit{5}}}\\
	\texttt{[0, 1, 4, 9, 16, 25]}
\end{flushleft}


\subsection*{Q1 – Solution}
\inputminted{python}{Files/11/1.py}

\subsection*{Q2: Range with Step Using Comprehension}
Prompt the user for \texttt{start} and \texttt{end}. Create a list from \texttt{start} to \texttt{end} (inclusive) with step 2 using a comprehension.

\textbf{Example:}
\begin{flushleft}
	\texttt{start = \textcolor{blue}{\textit{4}}}\\
	\texttt{end = \textcolor{blue}{\textit{10}}}\\
	\texttt{[4, 6, 8, 10]}
\end{flushleft}


\subsection*{Q2 – Solution}
\inputminted{python}{Files/11/2.py}


\subsection*{Q3: Numbers Not Divisible by 3}
Prompt the user for \texttt{N}. Create a list from 1 to \texttt{N} that excludes numbers divisible by 3.

\textbf{Example:}
\begin{flushleft}
	\texttt{N = \textcolor{blue}{\textit{10}}}\\
	\texttt{[1, 2, 4, 5, 7, 8, 10]}
\end{flushleft}

\subsection*{Q3 – Solution}
\inputminted{python}{Files/11/3.py}

\subsection*{Q4: Capitalized Words in a Sentence}
Ask the user to input a sentence. Use a list comprehension to extract only words that start with an uppercase letter.

\textbf{Example:}
\begin{flushleft}
	\texttt{Input: \textcolor{blue}{\textit{In einem rechtwinkligen Dreieck ist die Summe der Kathetenquadrate gleich dem Quadrat der Hypotenuse}}}\\
	\texttt{['In', 'Dreieck', 'Summe', 'Kathetenquadrate', 'Quadrat', 'Hypotenuse']}
\end{flushleft}

\subsection*{Q4 – Solution}
\inputminted{python}{Files/11/4.py}

\subsection*{Q5: Invert Dictionary with Comprehension}
Given the dictionary:

\begin{minted}{python}
	tab = {'Montag': 'Monday', 'Dienstag': 'Tuesday', 'Mittwoch': 'Wednesday',
		'Donnerstag': 'Thursday', 'Freitag': 'Friday', 'Samstag': 'Saturday',
		'Sonntag': 'Sunday'}
\end{minted}

Use a comprehension to create a new dictionary where keys and values are swapped.

\textbf{Expected Output:}
\begin{flushleft}
	\texttt{\{'Monday': 'Montag', 'Tuesday': 'Dienstag', ...\}}
\end{flushleft}

\subsection*{Q5 – Solution}
\inputminted{python}{Files/11/5.py}

\subsection*{Q6: Filter Even Month Numbers}
Given the dictionary:

\begin{minted}{python}
	tab = {'Jan': '1', 'Feb': '2', 'Mär': '3', 'Apr': '4', 'Mai': '5',
		'Jun': '6', 'Jul': '7', 'Aug': '8', 'Sep': '9', 'Okt': '10',
		'Nov': '11', 'Dez': '12'}
\end{minted}

Create a new dictionary using comprehension that includes only months with even numbers (converted to \texttt{int}).

\subsection*{Q6 – Solution}
\inputminted{python}{Files/11/6.py}

\textbf{Expected Output:}
\begin{flushleft}
	\texttt{\{'Feb': 2, 'Apr': 4, 'Jun': 6, 'Aug': 8, 'Okt': 10, 'Dez': 12\}}
\end{flushleft}

\subsection*{Q7: Set Comprehension from User Input}
Use a set comprehension to prompt the user five times for input. Store all unique values in a set. The full program should be no more than 2 lines.

\textbf{Example:}
\begin{flushleft}
	\texttt{Input: \textcolor{blue}{\textit{hallo}}}\\
	\texttt{Input: \textcolor{blue}{\textit{welt}}}\\
	\texttt{Input: \textcolor{blue}{\textit{hallo}}}\\
	\texttt{Input: \textcolor{blue}{\textit{berlin}}}\\
	\texttt{Input: \textcolor{blue}{\textit{hallo}}}\\
	\texttt{\{'hallo', 'welt', 'berlin'\}}
\end{flushleft}

\subsection*{Q7 – Solution}
\inputminted{python}{Files/11/7.py}

\newpage
\section{File Exercises}

\subsection*{Q1: File Analysis – Count Lines, Words, and Characters}
Write a program that reads a file named \texttt{automobil.txt}. The program should count the number of lines, the number of words, and the number of characters, and display them.



\textbf{Example Output:}
\begin{flushleft}
	\texttt{Number of lines: \textcolor{blue}{\textit{114}}}\\
	\texttt{Number of words: \textcolor{blue}{\textit{1735}}}\\
	\texttt{Number of characters: \textcolor{blue}{\textit{11807}}}
\end{flushleft}


\subsection*{Q1 – Solution}
\inputminted{python}{Files/12/1.py}

\subsection*{Q2: File Uppercase Conversion}
Write a program that reads the file \texttt{automobil.txt}, converts its content to uppercase, and writes the result into a new file named \texttt{result.txt}.


\subsection*{Q2 – Solution}
\inputminted{python}{Files/12/2.py}


\subsection*{Q3: Binary File Comparison}
Read two binary files \texttt{test1.bin} and \texttt{test2.bin}, compare them byte by byte, and count how many bytes differ. Output the result.

\textbf{Example Output:}
\begin{flushleft}
	\texttt{Differences: \textcolor{blue}{\textit{1216}}}
\end{flushleft}


\subsection*{Q3 – Solution}
\inputminted{python}{Files/12/3.py}

\newpage
\section{Exceptions}

\subsection*{Q1: Reciprocal Calculation with ZeroDivisionError}
Prompt the user to enter an integer \texttt{x}. Compute the reciprocal \texttt{1/x} and display the result. If the user enters 0, catch the \texttt{ZeroDivisionError} and print an appropriate message.

\textbf{Example 1:}
\begin{flushleft}
	\texttt{x = \textcolor{blue}{\textit{5}}}\\
	\texttt{1/x = 0.2}
\end{flushleft}

\textbf{Example 2:}
\begin{flushleft}
	\texttt{x = \textcolor{blue}{\textit{0}}}\\
	\texttt{Division by 0 not allowed}
\end{flushleft}

\subsection*{Q1 – Solution}
\inputminted{python}{Files/13/1.py}

\subsection*{Q2: Error Handling for Division and Type}
Extend the previous program. If the user inputs a word (not an integer), catch a \texttt{ValueError} and print an appropriate message.

\textbf{Example 1:}
\begin{flushleft}
	\texttt{x = \textcolor{blue}{\textit{5}}}\\
	\texttt{1/x = 0.2}
\end{flushleft}

\textbf{Example 2:}
\begin{flushleft}
	\texttt{x = \textcolor{blue}{\textit{0}}}\\
	\texttt{Division by 0 not allowed}
\end{flushleft}

\textbf{Example 3:}
\begin{flushleft}
	\texttt{x = \textcolor{blue}{\textit{hello}}}\\
	\texttt{Please enter a number}
\end{flushleft}


\subsection*{Q2 – Solution}
\inputminted{python}{Files/13/2.py}


\subsection*{Q3: Raising a ValueError for Negative Input}
Raise a \texttt{ValueError} when the user enters a negative number. All exceptions (\texttt{ZeroDivisionError}, \texttt{ValueError}) should be handled with a single \texttt{except} block.

\textbf{Example:}
\begin{flushleft}
	\texttt{x = \textcolor{blue}{\textit{-1}}}\\
	\texttt{Negative numbers not allowed}
\end{flushleft}

\subsection*{Q3 – Solution}
\inputminted{python}{Files/13/3.py}
	
\end{document}

