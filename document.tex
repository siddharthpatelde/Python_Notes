\documentclass[a4paper,11pt]{article}

% ───────────────────────────
%         PACKAGES
% ───────────────────────────
\usepackage[utf8]{inputenc}
\usepackage[T1]{fontenc}
\usepackage{geometry}
\usepackage{graphicx}
\usepackage{amsmath, amssymb}
\usepackage{booktabs}
\usepackage{enumitem}
\usepackage{hyperref}
\usepackage{caption}
\usepackage{subcaption}

\usepackage{fancyhdr}
\usepackage{titlesec}
\usepackage{parskip}

\usepackage{minted}
\usepackage{inconsolata}  % clean monospaced font
\usepackage{mathpazo}     % professional text font

\usepackage{array}
\usepackage{booktabs}
\usepackage[table]{xcolor}


% ───────────────────────────
%         GEOMETRY
% ───────────────────────────
\geometry{
	a4paper,
	left=2.5cm,
	right=2.5cm,
	top=2.5cm,
	bottom=2.5cm
}

% ───────────────────────────
%         HEADER/FOOTER
% ───────────────────────────
\pagestyle{fancy}
\fancyhf{}
\renewcommand{\headrulewidth}{0.4pt}
\renewcommand{\footrulewidth}{0.4pt}
\lhead{\textsf{\textcolor{gray}{Python Notes}}}
\rhead{\textsf{\textcolor{gray}{Siddharth Patel}}}
\rfoot{\textsf{\textcolor{gray}{\thepage}}}

% ───────────────────────────
%         TITLE FORMAT
% ───────────────────────────
\titleformat{\section}
{\Large\bfseries\sffamily\color{black}}
{\thesection}{1em}{}

\titleformat{\subsection}
{\large\bfseries\sffamily\color{black}}
{\thesubsection}{1em}{}

% ───────────────────────────
%         MINTED SETTINGS
% ───────────────────────────
\definecolor{mintbg}{rgb}{0.97,0.97,0.97}
\setminted{
	bgcolor=mintbg,
	fontsize=\footnotesize,
	linenos=true,
	numbersep=5pt,
	frame=lines,
	framesep=2mm,
	breaklines=true,
	tabsize=4,
	encoding=utf8
}



% ───────────────────────────
%         TITLE PAGE
% ───────────────────────────
\title{\Huge \textbf{My Python Course Notes}\\[1ex]
	\Large Structured Revision for Every Lesson}
\author{\textbf{Siddharth Patel} \\ HTW Berlin}
\date{\today}

% ───────────────────────────
%         BEGIN DOCUMENT
% ───────────────────────────
\begin{document}
	
	\maketitle
	\thispagestyle{empty}
	\newpage
	\tableofcontents
	\newpage
	
	% ───────────────────────────
	%         FIRST LESSON
	% ───────────────────────────
	\section{Lesson 1: Print Function – Full Usage Guide}
	\inputminted{python}{Python_Files/print_guid.py}
	
	\vspace{1em}
	\subsection*{Additional Functions Used in This Lesson}
	
\vspace{1em}
\subsection*{Referenced Functions – Syntax and Output Type}

\rowcolors{2}{gray!10}{white}
\begin{tabular}{>{\bfseries}p{4cm} p{7.5cm} p{3cm}}
	\toprule
	Function & Syntax & Return / Output Type \\
	\midrule
	
	\texttt{with open()} & \texttt{with open("file.txt", "w") as f:} & File object \\
	\texttt{print(..., file=f)} & \texttt{print("text", file=f)} & Writes to file, returns \texttt{None} \\
	
	\texttt{range()} & \texttt{range(3)} or \texttt{range(start, stop, step)} & Range object (iterable) \\
	
	\texttt{time.sleep()} & \texttt{time.sleep(seconds)} & None (pauses execution) \\
	
	\bottomrule
\end{tabular}

	
	
	\section{Lesson 2: Input Function – Full Usage Guide}
	\inputminted{python}{Python_Files/input_guid.py}
	
	
	\vspace{1em}
	\subsection*{Referenced Functions – Syntax and Output Type}
	
	\rowcolors{2}{gray!10}{white}
	\begin{tabular}{>{\bfseries}p{3.5cm} p{8cm} p{3cm}}
		\toprule
		Function / Statement & Syntax & Return / Output Type \\
		\midrule
		
		\texttt{.split()} & \texttt{string.split()} or \texttt{string.split("delimiter")} & List of strings \\
		
		\texttt{map()} & \texttt{map(function, iterable)} & Map object (can be converted to list) \\
		
		\texttt{list()} & \texttt{list(iterable)} & List object \\
		
		\texttt{try / except} & 
		\texttt{try:\newline \hspace{1em} code\newline except ErrorType:\newline \hspace{1em} fallback} & Flow control – no return value; handles runtime errors \\
		
		\bottomrule
	\end{tabular}
	
	
	
	
	\section{Lesson 3: Math Operators – Full Usage Guide}
	\inputminted{python}{Python_Files/math_operators_guid.py}
	
	\section{Lesson 4: Strings – Full Usage Guide}
	\inputminted{python}{Python_Files/string_guid.py}
	
	\vspace{1em}
	\subsection*{Referenced Methods – Syntax and Output Type}
	
	\rowcolors{2}{gray!10}{white}
	\begin{tabular}{>{\bfseries}p{3.5cm} p{8cm} p{3cm}}
		\toprule
		Method / Function & Syntax & Return / Output Type \\
		\midrule
		
		\texttt{.capitalize()} & \texttt{str.capitalize()} & \texttt{str} \\
		
		\texttt{.lower()} & \texttt{str.lower()} & \texttt{str} \\
		
		\texttt{.title()} & \texttt{str.title()} & \texttt{str} \\
		
		\texttt{.casefold()} & \texttt{str.casefold()} & \texttt{str} \\
		
		\texttt{.upper()} & \texttt{str.upper()} & \texttt{str} \\
		
		\texttt{.count()} & \texttt{str.count(substring, start, end)} & \texttt{int} \\
		
		\texttt{.find()} & \texttt{str.find(substring, start, end)} & \texttt{int} \\
		
		\texttt{.replace()} & \texttt{str.replace(old, new, count)} & \texttt{str} \\
		
		\texttt{.swapcase()} & \texttt{str.swapcase()} & \texttt{str} \\
		
		\texttt{.join()} & \texttt{"separator".join(iterable)} & \texttt{str} \\
		
		\bottomrule
	\end{tabular}
	
	
	\section{Lesson 5: If, Else, and Conditional Operators}
	\inputminted{python}{Python_Files/if_else_guid.py}
	
	\vspace{1em}
	\subsection*{Referenced Operators – Syntax and Output Type}
	
	\rowcolors{2}{gray!10}{white}
	\begin{tabular}{>{\bfseries}p{4cm} p{7.5cm} p{3cm}}
		\toprule
		Operator & Syntax & Return / Output Type \\
		\midrule
		
		\texttt{==} (Equal) & \texttt{x == y} & \texttt{bool} \\
		
		\texttt{!=} (Not Equal) & \texttt{x != y} & \texttt{bool} \\
		
		\texttt{<} (Less Than) & \texttt{x < y} & \texttt{bool} \\
		
		\texttt{<=} (Less Than or Equal) & \texttt{x <= y} & \texttt{bool} \\
		
		\texttt{>} (Greater Than) & \texttt{x > y} & \texttt{bool} \\
		
		\texttt{>=} (Greater Than or Equal) & \texttt{x >= y} & \texttt{bool} \\
		
		\texttt{and} (Logical AND) & \texttt{x > 5 and x < 10} & \texttt{bool} \\
		
		\texttt{or} (Logical OR) & \texttt{x < 5 or x > 10} & \texttt{bool} \\
		
		\texttt{not} (Logical NOT) & \texttt{not (x > 5)} & \texttt{bool} \\
		
		\textit{Ternary Expression} & \texttt{value1 if condition else value2} & Result of \texttt{value1} or \texttt{value2} \\
		\bottomrule
	\end{tabular}
	
	
	
	
	
\end{document}
